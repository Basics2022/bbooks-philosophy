%% Generated by Sphinx.
\def\sphinxdocclass{jupyterBook}
\documentclass[letterpaper,10pt,italian]{jupyterBook}
\ifdefined\pdfpxdimen
   \let\sphinxpxdimen\pdfpxdimen\else\newdimen\sphinxpxdimen
\fi \sphinxpxdimen=.75bp\relax
\ifdefined\pdfimageresolution
    \pdfimageresolution= \numexpr \dimexpr1in\relax/\sphinxpxdimen\relax
\fi
%% let collapsible pdf bookmarks panel have high depth per default
\PassOptionsToPackage{bookmarksdepth=5}{hyperref}
%% turn off hyperref patch of \index as sphinx.xdy xindy module takes care of
%% suitable \hyperpage mark-up, working around hyperref-xindy incompatibility
\PassOptionsToPackage{hyperindex=false}{hyperref}
%% memoir class requires extra handling
\makeatletter\@ifclassloaded{memoir}
{\ifdefined\memhyperindexfalse\memhyperindexfalse\fi}{}\makeatother

\PassOptionsToPackage{warn}{textcomp}

\catcode`^^^^00a0\active\protected\def^^^^00a0{\leavevmode\nobreak\ }
\usepackage{cmap}
\usepackage{fontspec}
\defaultfontfeatures[\rmfamily,\sffamily,\ttfamily]{}
\usepackage{amsmath,amssymb,amstext}
\usepackage{polyglossia}
\setmainlanguage{italian}



\setmainfont{FreeSerif}[
  Extension      = .otf,
  UprightFont    = *,
  ItalicFont     = *Italic,
  BoldFont       = *Bold,
  BoldItalicFont = *BoldItalic
]
\setsansfont{FreeSans}[
  Extension      = .otf,
  UprightFont    = *,
  ItalicFont     = *Oblique,
  BoldFont       = *Bold,
  BoldItalicFont = *BoldOblique,
]
\setmonofont{FreeMono}[
  Extension      = .otf,
  UprightFont    = *,
  ItalicFont     = *Oblique,
  BoldFont       = *Bold,
  BoldItalicFont = *BoldOblique,
]



\usepackage[Sonny]{fncychap}
\ChNameVar{\Large\normalfont\sffamily}
\ChTitleVar{\Large\normalfont\sffamily}
\usepackage[,numfigreset=1,mathnumfig]{sphinx}

\fvset{fontsize=\small}
\usepackage{geometry}


% Include hyperref last.
\usepackage{hyperref}
% Fix anchor placement for figures with captions.
\usepackage{hypcap}% it must be loaded after hyperref.
% Set up styles of URL: it should be placed after hyperref.
\urlstyle{same}


\usepackage{sphinxmessages}



        % Start of preamble defined in sphinx-jupyterbook-latex %
         \usepackage[Latin,Greek]{ucharclasses}
        \usepackage{unicode-math}
        % fixing title of the toc
        \addto\captionsenglish{\renewcommand{\contentsname}{Contents}}
        \hypersetup{
            pdfencoding=auto,
            psdextra
        }
        % End of preamble defined in sphinx-jupyterbook-latex %
        

\title{basics - philosophy}
\date{21 dic 2024}
\release{}
\author{basics}
\newcommand{\sphinxlogo}{\vbox{}}
\renewcommand{\releasename}{}
\makeindex
\begin{document}

\pagestyle{empty}
\sphinxmaketitle
\pagestyle{plain}
\sphinxtableofcontents
\pagestyle{normal}
\phantomsection\label{\detokenize{intro::doc}}


\sphinxAtStartPar
Please check out the Github repo of the project, \sphinxhref{https://github.com/Basics2022}{basics\sphinxhyphen{}book project}.

\sphinxAtStartPar
Questo materiale nasce come appunti personali, riassunto di una storia parziale della filosofia.
\begin{itemize}
\item {} 
\sphinxAtStartPar
{\hyperref[\detokenize{ch/history::doc}]{\sphinxcrossref{Cronologia filosofia occiedentale}}}

\item {} 
\sphinxAtStartPar
{\hyperref[\detokenize{ch/topics::doc}]{\sphinxcrossref{Argomenti}}}

\end{itemize}

\sphinxAtStartPar
Viene posta attenzione alla filosofia occidentale moderna e alle sue relazioni con la scienza naturale e la scienza sociale. Un breve riassunto della filosofia antica sarà inevitabile, per i riferimenti a essa da parte della filosofia moderna. Per provare a lasciare meno buchi possibili senza perdersi negli argomenti più morali e religiosi, sarà necessaria una sintesi estrema della filosofia medievale.

\sphinxstepscope


\chapter{Cronologia filosofia occiedentale}
\label{\detokenize{ch/history:cronologia-filosofia-occiedentale}}\label{\detokenize{ch/history:philosophy-chronology}}\label{\detokenize{ch/history::doc}}\begin{itemize}
\item {} 
\sphinxAtStartPar
Filosofia antica

\item {} 
\sphinxAtStartPar
Filosofia medievale

\item {} 
\sphinxAtStartPar
Filosofia moderna
\begin{itemize}
\item {} 
\sphinxAtStartPar
Neoplatonismo

\item {} 
\sphinxAtStartPar
Naturalismo, Razionalismo, Materialismo, Meccanicismo, Empirismo

\item {} 
\sphinxAtStartPar
Illuminismo

\item {} 
\sphinxAtStartPar
Idealismo: Fichte, Shelling, Hegel

\item {} 
\sphinxAtStartPar
Materialismo: Marx

\item {} 
\sphinxAtStartPar
Esistenzialismo

\item {} 
\sphinxAtStartPar
Positivismo

\end{itemize}

\item {} 
\sphinxAtStartPar
Filosofia del XX secolo
\begin{itemize}
\item {} 
\sphinxAtStartPar
Filosofia analitica: Russell, Moore, Wittgenstein; positivismo logico; empirismo logico

\item {} 
\sphinxAtStartPar
Filosofia continentale: psicoanalisi (?), marxismo, fenomenologia, esistenzialismo, post\sphinxhyphen{}strutturalismo, post\sphinxhyphen{}modernismo, decostruzionismo, teoria critica,…

\end{itemize}

\end{itemize}


\section{Romanticismo}
\label{\detokenize{ch/history:romanticismo}}

\section{Illuminismo}
\label{\detokenize{ch/history:illuminismo}}\label{\detokenize{ch/history:pc-illuminism}}
\sphinxAtStartPar
\sphinxstylestrong{Personaggi.}
\begin{itemize}
\item {} 
\sphinxAtStartPar
Kant(Konigsberg, 1724\sphinxhyphen{}1804): critiche

\end{itemize}


\section{Idealismo}
\label{\detokenize{ch/history:pc-idealism}}\label{\detokenize{ch/history:idealismo}}

\section{Liberismo}
\label{\detokenize{ch/history:liberismo}}\label{\detokenize{ch/history:pc-liberism}}
\sphinxAtStartPar
Economisti classici: A.Smith \sphinxhyphen{}> illuminismo, D.Ricardo


\section{Positivismo}
\label{\detokenize{ch/history:positivismo}}\label{\detokenize{ch/history:pc-positivism}}
\sphinxAtStartPar
\sphinxstylestrong{Contesto.} Sviluppo nella prima metà dell’XIX secolo in Francia e Gran Bretagna, diffusione nella seconda metà del secolo. Il secolo è caratterizzato dalla rivoluzione industriale.

\sphinxAtStartPar
\sphinxstylestrong{Legami.}
\begin{itemize}
\item {} 
\sphinxAtStartPar
Illuminismo: fiducia nella scienza; ma illuminismo critico nei confronti dell’esistente, gran differenza tra il mondo com’è e il mondo come dovrebbe essere (Kant, Voltaire, Rousseau); i filosofi positivisti invece sono allineati al pensiero dominante del tempo: la prima parte del positivismo è conservatrice, alllineata con i sovrani assoluti; la seconda parte è espressione della classe borghese, come classe dominante: pensiero dei risultati raggiunti

\end{itemize}

\sphinxAtStartPar
\sphinxstylestrong{Argomenti.}
\begin{itemize}
\item {} 
\sphinxAtStartPar
valorizzazione del progresso e della scienza, come unica forma di conoscenza valida, attraverso il metodo scientifico;

\item {} 
\sphinxAtStartPar
estensione del metodo scientifico ad ambiti sociali, come in logica, psicologia, sociologia, economia

\item {} 
\sphinxAtStartPar
rifiuto della metafisica, poiché metodo scientifico inapplicabile

\item {} 
\sphinxAtStartPar
a cosa serve la filosofia? Necessaria riforma: abbandonare studi di metafisica, abbracciare tutti gli argomenti trattati se applicabile il metodo scientifico; una volta che una disciplina è una disciplina scientifica, non ci sono discipline privilegiate. La filosofia come metodi comuni a tutte le scienze, attività di coordinamento e guida, in un desiderio di unificazione.

\end{itemize}

\sphinxAtStartPar
Due periodi:
\begin{itemize}
\item {} 
\sphinxAtStartPar
primo positivismo: sviluppatosi in Francia, Comte. Subisce ancora la forte influenza dell’idealismo. Filosofia progressista dal punto di vista scientifico, conservatrice dal punto di vista politico (contraria ai moti)

\item {} 
\sphinxAtStartPar
secondo positivismo: progressismo dal punto di vista scientifico e politico, filosofia liberale, che caratterizza il ceto borghese

\end{itemize}

\sphinxAtStartPar
Altri aspetti
\begin{itemize}
\item {} 
\sphinxAtStartPar
positivismo sociale: Comte, sociologia

\item {} 
\sphinxAtStartPar
positivismo (metodo)logico: Stuart Mill, si concentra sul metodo per fare scienza e conoscenza

\item {} 
\sphinxAtStartPar
positivismo evoluzionistico: influenzato da Darwin, filosofi prenderanno le osservazioni di Darwin sull’evoluzione per farne una legge di natura e applicarla agli ambiti della filosofia

\end{itemize}

\sphinxAtStartPar
\sphinxstylestrong{Personaggi.}

\phantomsection\label{\detokenize{ch/history:pc-saint-simon}}\subsubsection*{Saint\sphinxhyphen{}Simon (Parigi, 1760 \sphinxhyphen{} 1825)}

\sphinxAtStartPar
\sphinxstylestrong{Biografia.}
\sphinxstylestrong{Opere.}
\sphinxstylestrong{Pensiero.}
\begin{itemize}
\item {} 
\sphinxAtStartPar
Storia tende naturalmente al progresso, ma segnata da alternanza di \sphinxstyleemphasis{epoche critiche} ed \sphinxstyleemphasis{epoche organiche}. A inizio “800 è finita l’epoca critica della nascita di un nuova scienza e della rivoluzione industriale, ora che si sono imposte la storia e l’umanità può evolvere in maniera costante

\item {} 
\sphinxAtStartPar
Scontro tra ceti produttivi e parassiti, i ceti improduttivi: padroni e operai dovrebbero allearsi contro nobili, clero, burocrati che vivono sulle spalle dei lavoratori

\end{itemize}
\phantomsection\label{\detokenize{ch/history:pc-comte}}\subsubsection*{Comte (Montpellier, 1798 \sphinxhyphen{} Parigi, 1857)}

\sphinxAtStartPar
\sphinxstylestrong{Biografia} Allievo (\sphinxstylestrong{todo} \sphinxstyleemphasis{di che?}) di {\hyperref[\detokenize{ch/history:pc-saint-simon}]{\sphinxcrossref{\DUrole{std,std-ref}{Saint\sphinxhyphen{}Simon}}}}. Crollo psicologico…
\sphinxstylestrong{Opere.} 1830, \sphinxstyleemphasis{Corso di filosofia positiva}
\sphinxstylestrong{Pensiero.}
\begin{itemize}
\item {} 
\sphinxAtStartPar
La scienza ha guidato la storia e può continuare a guidare la storia. Il metodo scientifico può essere applicato a tutti gli ambiti della vita.

\item {} 
\sphinxAtStartPar
\sphinxstylestrong{Legge dei tre stadi.} Attraverso lo studio della storia delle scienza (astronomia, fisica\sphinxhyphen{}meccanica, chimica, biologia) propone una legge di evoluzione delle scienze attraverso tre stadi:
\begin{enumerate}
\sphinxsetlistlabels{\arabic}{enumi}{enumii}{}{.}%
\item {} 
\sphinxAtStartPar
stadio teologico o fittizio: l’uomo usa l’immaginazione per dare una spiegazione tramite entità immaginarie soprannaturali

\item {} 
\sphinxAtStartPar
stadio metafisico: astrazioni personificate

\item {} 
\sphinxAtStartPar
stadio positivo o scientifico: conoscenze tramite fatti e relazioni

\end{enumerate}

\item {} 
\sphinxAtStartPar
C’è ancora una disciplina che non ha ancora compiuto l’ultimo passo: la \sphinxstylestrong{sociologia}. Necessità di compiere questo passo per comprendere, \sphinxstylestrong{prevedere} e controllare le dinamiche della società, ad esempio per evitare crisi. Visto che queste leggi non esistono, si procede «a tentoni». Lo scopo della filosofia di Comte è portare la sociologia a compiere questo ultimo passo, formulando una \sphinxstyleemphasis{scienza matura, con un numero limitato di leggi}. Con queste leggi Comte non si propone di capire il senso intimo della vita: non ci si occupa di entità astratte (noumeno come realtà profonda delle cose, del fenomeno, come Kant), ma solo di fatti concreti e misurabili, il resto sono parole

\item {} 
\sphinxAtStartPar
Mancano all’appello 3 discipline: matematica (linguaggio della scienza), logica (grammatica della scienza), psicologia. La psicologia deve essere divisa in due parti: gli argomenti sono riconducibili o alla biologia o alla sociologia.

\item {} 
\sphinxAtStartPar
\sphinxstylestrong{Sociologia.} Comte distingue due discipline all’interno della sociologia:
\begin{itemize}
\item {} 
\sphinxAtStartPar
statica sociale: fotografia dell’istante; a ogni sistema politico corrisponde un sistema economico e sociale; la statica si occupa dell”\sphinxstylestrong{ordine} di un tipo di società

\item {} 
\sphinxAtStartPar
dinamica sociale: cambiamenti, \sphinxstylestrong{progresso}; per Comte la \sphinxstylestrong{storia} è una evoluzione che segue un \sphinxstylestrong{percorso necessario}, come Hegel
\begin{itemize}
\item {} 
\sphinxAtStartPar
per Hegel la storia è guidata dalla provvidenza, l’assoluto, una divinità spirituale; per Comte la storia è guidata dalla scienza

\item {} 
\sphinxAtStartPar
chi porta avanti la scienza: uomini di genio; per Hegel, eroi o veggenti; per Comte, gli uomini di scienza

\end{itemize}

\end{itemize}

\end{itemize}
\phantomsection\label{\detokenize{ch/history:pc-bentham}}\subsubsection*{Bentham (Londra, 1748 \sphinxhyphen{} 1832)}

\sphinxAtStartPar
\sphinxstylestrong{Pensiero.} Ispirandosi a C.Beccaria (Milano, 1738\sphinxhyphen{}1794), bisogna occuparsi di portare il maggior bene possibile al numero di persone maggiore possibile. \sphinxstylestrong{Utilitarismo}: bisogna muoversi secondo l’obiettivo di Beccaria, «misurando il bene», usando un insieme di parametri: prossimità, intensità, durata, fecondità,…Valutando piacere e dolore secondo questi parametri, si può valutare l’utilità delle azioni, distinguendo quelle virtuose da quelle dannosi.

\sphinxAtStartPar
J.Mill, padre di \DUrole{xref,myst}{Stuard Mill} applica queste idee alla psicologia. Condizionamento da esperienze precedenti, favorisce un comportamento morale in senso utilitarista
\phantomsection\label{\detokenize{ch/history:pc-stuart-mill}}\subsubsection*{Stuart Mill (1806 \sphinxhyphen{} 1873)}

\sphinxAtStartPar
\sphinxstylestrong{Biografia.} Moglie H.Taylor, contribuisce alle idee del marito.

\sphinxAtStartPar
\sphinxstylestrong{Pensiero.} Utilitarista, positivista, liberalista. Pensatore politico
\begin{itemize}
\item {} 
\sphinxAtStartPar
1859 \sphinxstyleemphasis{On Liberty}, Stuart Mill presenta la sua idea di libertà. Il progresso si è fondato sulla diversità di opinione. Spesso un pensatore controcorrente ha portato progresso. Necessità di preservare la \sphinxstylestrong{libertà di opinione}. Minacce alla libertà di opinione: l’industria ha portato a un livellamento di consumi, omologazione; l’istruzione pubblica rischia di uniformare il pensiero; la politica parlamentare dà l’illusione di essere tutti uguali, ma c’è bisogno di differenza, di pluralità. C’è il rischio della \sphinxstyleemphasis{dittatura della maggioranza} (A. de Tocqueville): bisogna difendere anche chi la pensa diversamente, al costo di ascoltare opinioni estreme o assurde. Libertà di espressione (ma non di azione! Libertà di azione, a meno che limiti la libertà altrui) e controbattito. Anche le idee più assurde possono servire a rivedere le proprie opinioni. Esempio sulla forza di gravità: noi non ci facciamo caso, ma la percepiamo; speigarne l’esistenza a chi non ne crede può aiutare a rinforzare le proprie tesi, migliorare la nostra comprensione,…
Due esempi in cui la maggioranza non ha avuto ragione: morte di Socrate e di Gesù.

\end{itemize}

\sphinxAtStartPar
Libertà di azione, a meno che limiti la libertà di altri: contrario a legge vigente in UK contro l’omosessualità, contrario a proibizionismo alcolici,…

\sphinxAtStartPar
Spinta verso riconoscimento diritti delle donne: si sta escludendo metà della popolazione del genere umano dalla vita sociale (no diritto di voto, necessità di tutore per amministrazione beni,…), nell’ambito del pluralismo. Le donne potevano essere meno capaci ad aministrare denaro, ma quello è un problema educativo non strutturale nella natura delle donne. Esempi: regine di Inghilterra, Elisabetta I (1533\sphinxhyphen{}1603) e Vittoria (1819\sphinxhyphen{}1901). Proposta di estensione diritti, matrimonio paritetico (nei matrimoni di vecchio stampo, le donne subivano un ricatto affettivo in famiglia; suddivisione compiti domestici, per liberare le donne dal ricatto affettivo di famiglia; i figli avranno educazione plurale, e quindi meglio),…
\phantomsection\label{\detokenize{ch/history:pc-darwin}}\subsubsection*{Darwin (Shrewsbury 1802 \sphinxhyphen{} Downe, 1882)}

\sphinxAtStartPar
Non propriamente un filosofo, ma le sue idee hanno forte influenza.

\sphinxAtStartPar
\sphinxstylestrong{Pensiero.} Selezione naturale. Tutte le specie sono soggette a piccole mutazioni casuali; alcune di queste mutazioni \sphinxhyphen{} fino a un certo punto ininfluenti \sphinxhyphen{} possono rispondere meglio alle nuove esigenze createsi in seguito a un cambiamento improvviso dell’ambiente; nell’arco di alcune generazioni, gli individui con i geni «più adatti alla riproduzione» prosperano e questo dà origine all’evoluzione della specie. Osservazione in:
\begin{itemize}
\item {} 
\sphinxAtStartPar
osservazioni bilogiche

\item {} 
\sphinxAtStartPar
osservazioni economiche? Malthus

\end{itemize}
\phantomsection\label{\detokenize{ch/history:pc-spencer}}\subsubsection*{Spencer (Derby, 1820 \sphinxhyphen{} Brighton, 1903)}

\sphinxAtStartPar
\sphinxstylestrong{Biografia.} Ingegnere in ambito ferroviario.

\sphinxAtStartPar
\sphinxstylestrong{Opere}
\begin{itemize}
\item {} 
\sphinxAtStartPar
1860,

\end{itemize}

\sphinxAtStartPar
\sphinxstylestrong{Pensiero.} Religione e scienza non sono antitetiche. La scienza si occupa di tutto quello che c’è dopo un limite, poiché non è in grado di spiegare la causa prima di quello che osserviamo.

\sphinxAtStartPar
Qual è il ruolo della filosofia? Unificare religione e scienza, sotto la legge dell’evoluzione. Le scienze sono pervenute a 3 principi fondamentali:
\begin{itemize}
\item {} 
\sphinxAtStartPar
industrittibilità della \sphinxstylestrong{materia}

\item {} 
\sphinxAtStartPar
continuità del \sphinxstylestrong{movimento}

\item {} 
\sphinxAtStartPar
persistenza della \sphinxstylestrong{forza}
Con questi tre principi, Spencer pensa di aver formulato la \sphinxstylestrong{legge dell’evoluzione}: tutto procede da una situazione di minor coerenza a una di maggior coerenza (?), da una maggior omogeneità verso l’eterogeneità. Esempi: universo, società;…

\end{itemize}

\sphinxAtStartPar
\sphinxstyleemphasis{Sopravvivenza del più adatto}: da un discorso biologico si passa a un discorso sociale, morale. Secondo Spencer, il futuro non può essere cambiato; ogni aspetto della vita evolve verso il bene seguendo la legge di evoluzione. Quindi, Spencer critica Comte e il suo desiderio di formulare le leggi della sociologia per poi prevedere l’evoluzione della società e intervenire. Per Spencer qualsiasi intervento è inutile (chi non ce la fa, pazienza, è inadatto e destinato a sparire). Questo pensiero ha un essito giustificazionista, l’evoluzione viaggia sempre verso il meglio, non c’è bisogno di cambiarlo, non c’è bisogno di criticarlo, non c’è bisogno di rivoluzioni…

\sphinxAtStartPar
Tre fasi della storia umana: 1. militare, 2. industriale, 3. futura nella quale gli individui cureranno al proprio individualismo e coopereranno per il proprio bene.

\sphinxAtStartPar
\sphinxstyleemphasis{(tutti principi che in un modo o nell’altro sono imprecisi o vengono sconvolti dalla fisica del XX secolo)}
\sphinxstyleemphasis{(legge verso la diversficazione? Forse qualcosa di simile all’entropia…)}

\sphinxstepscope


\chapter{Argomenti}
\label{\detokenize{ch/topics:argomenti}}\label{\detokenize{ch/topics:philosophy-topics}}\label{\detokenize{ch/topics::doc}}

\section{Ontologia}
\label{\detokenize{ch/topics:ontologia}}\begin{itemize}
\item {} 
\sphinxAtStartPar
Kant: fenomeno e noumeno

\item {} 
\sphinxAtStartPar
{\hyperref[\detokenize{ch/history:pc-comte}]{\sphinxcrossref{\DUrole{std,std-ref}{Comte}}}}: legge dei tre stadi

\end{itemize}


\section{Metafisica}
\label{\detokenize{ch/topics:metafisica}}

\section{Storia e sociologia}
\label{\detokenize{ch/topics:storia-e-sociologia}}\begin{itemize}
\item {} 
\sphinxAtStartPar
\DUrole{xref,myst}{Hegel}

\item {} 
\sphinxAtStartPar
{\hyperref[\detokenize{ch/history:pc-saint-simon}]{\sphinxcrossref{\DUrole{std,std-ref}{Saint\sphinxhyphen{}Simon}}}}

\item {} 
\sphinxAtStartPar
{\hyperref[\detokenize{ch/history:pc-comte}]{\sphinxcrossref{\DUrole{std,std-ref}{Comte}}}}

\end{itemize}


\section{Psicologia}
\label{\detokenize{ch/topics:psicologia}}\begin{itemize}
\item {} 
\sphinxAtStartPar
{\hyperref[\detokenize{ch/history:pc-comte}]{\sphinxcrossref{\DUrole{std,std-ref}{Comte}}}}: psicologia da smembrare e ricondurre a biologia o a \sphinxstylestrong{sociologia}

\end{itemize}







\renewcommand{\indexname}{Indice}
\printindex
\end{document}