%% Generated by Sphinx.
\def\sphinxdocclass{jupyterBook}
\documentclass[letterpaper,10pt,italian]{jupyterBook}
\ifdefined\pdfpxdimen
   \let\sphinxpxdimen\pdfpxdimen\else\newdimen\sphinxpxdimen
\fi \sphinxpxdimen=.75bp\relax
\ifdefined\pdfimageresolution
    \pdfimageresolution= \numexpr \dimexpr1in\relax/\sphinxpxdimen\relax
\fi
%% let collapsible pdf bookmarks panel have high depth per default
\PassOptionsToPackage{bookmarksdepth=5}{hyperref}
%% turn off hyperref patch of \index as sphinx.xdy xindy module takes care of
%% suitable \hyperpage mark-up, working around hyperref-xindy incompatibility
\PassOptionsToPackage{hyperindex=false}{hyperref}
%% memoir class requires extra handling
\makeatletter\@ifclassloaded{memoir}
{\ifdefined\memhyperindexfalse\memhyperindexfalse\fi}{}\makeatother

\PassOptionsToPackage{warn}{textcomp}

\catcode`^^^^00a0\active\protected\def^^^^00a0{\leavevmode\nobreak\ }
\usepackage{cmap}
\usepackage{fontspec}
\defaultfontfeatures[\rmfamily,\sffamily,\ttfamily]{}
\usepackage{amsmath,amssymb,amstext}
\usepackage{polyglossia}
\setmainlanguage{italian}



\setmainfont{FreeSerif}[
  Extension      = .otf,
  UprightFont    = *,
  ItalicFont     = *Italic,
  BoldFont       = *Bold,
  BoldItalicFont = *BoldItalic
]
\setsansfont{FreeSans}[
  Extension      = .otf,
  UprightFont    = *,
  ItalicFont     = *Oblique,
  BoldFont       = *Bold,
  BoldItalicFont = *BoldOblique,
]
\setmonofont{FreeMono}[
  Extension      = .otf,
  UprightFont    = *,
  ItalicFont     = *Oblique,
  BoldFont       = *Bold,
  BoldItalicFont = *BoldOblique,
]



\usepackage[Sonny]{fncychap}
\ChNameVar{\Large\normalfont\sffamily}
\ChTitleVar{\Large\normalfont\sffamily}
\usepackage[,numfigreset=1,mathnumfig]{sphinx}

\fvset{fontsize=\small}
\usepackage{geometry}


% Include hyperref last.
\usepackage{hyperref}
% Fix anchor placement for figures with captions.
\usepackage{hypcap}% it must be loaded after hyperref.
% Set up styles of URL: it should be placed after hyperref.
\urlstyle{same}


\usepackage{sphinxmessages}



        % Start of preamble defined in sphinx-jupyterbook-latex %
         \usepackage[Latin,Greek]{ucharclasses}
        \usepackage{unicode-math}
        % fixing title of the toc
        \addto\captionsenglish{\renewcommand{\contentsname}{Contents}}
        \hypersetup{
            pdfencoding=auto,
            psdextra
        }
        % End of preamble defined in sphinx-jupyterbook-latex %
        

\title{basics book template}
\date{21 dic 2024}
\release{}
\author{basics}
\newcommand{\sphinxlogo}{\vbox{}}
\renewcommand{\releasename}{}
\makeindex
\begin{document}

\pagestyle{empty}
\sphinxmaketitle
\pagestyle{plain}
\sphinxtableofcontents
\pagestyle{normal}
\phantomsection\label{\detokenize{intro::doc}}


\sphinxAtStartPar
If you want ot start a new basics\sphinxhyphen{}book, it could be a good idea to start from this template.

\sphinxAtStartPar
Please check out the Github repo of the project, \sphinxhref{https://github.com/Basics2022}{basics\sphinxhyphen{}book project}.
\begin{itemize}
\item {} 
\sphinxAtStartPar
{\hyperref[\detokenize{ch/history::doc}]{\sphinxcrossref{Cronologia}}}

\item {} 
\sphinxAtStartPar
{\hyperref[\detokenize{ch/topics::doc}]{\sphinxcrossref{Argomenti}}}

\end{itemize}

\sphinxstepscope


\chapter{Cronologia}
\label{\detokenize{ch/history:cronologia}}\label{\detokenize{ch/history:philosophy-chronology}}\label{\detokenize{ch/history::doc}}

\section{Illuminismo}
\label{\detokenize{ch/history:illuminismo}}\label{\detokenize{ch/history:pc-illuminism}}
\begin{sphinxVerbatim}[commandchars=\\\{\}]
\PYG{o}{*}\PYG{o}{*}\PYG{n}{Personaggi}\PYG{o}{.}\PYG{o}{*}\PYG{o}{*}
\PYG{o}{\PYGZhy{}} \PYG{n}{Kant}\PYG{p}{(}\PYG{n}{Ko} \PYG{n}{temponigsberg}\PYG{p}{,} \PYG{l+m+mi}{1724}\PYG{o}{\PYGZhy{}}\PYG{l+m+mi}{1804}\PYG{p}{)}\PYG{p}{:} \PYG{n}{critiche}

\end{sphinxVerbatim}


\section{Idealismo}
\label{\detokenize{ch/history:pc-idealism}}\label{\detokenize{ch/history:idealismo}}

\section{Positivismo}
\label{\detokenize{ch/history:positivismo}}\label{\detokenize{ch/history:pc-positivism}}
\sphinxAtStartPar
\sphinxstylestrong{Contesto.} Sviluppo nella prima metà dell’XIX secolo in Francia e Gran Bretagna, diffusione nella seconda metà del secolo. Il secolo è caratterizzato dalla rivoluzione industriale.

\sphinxAtStartPar
\sphinxstylestrong{Legami.}
\begin{itemize}
\item {} 
\sphinxAtStartPar
Illuminismo: fiducia nella scienza; ma illuminismo critico nei confronti dell’esistente, gran differenza tra il mondo com’è e il mondo come dovrebbe essere (Kant, Voltaire, Rousseau); i filosofi positivisti invece sono allineati al pensiero dominante del tempo: la prima parte del positivismo è conservatrice, alllineata con i sovrani assoluti; la seconda parte è espressione della classe borghese, come classe dominante: pensiero dei risultati raggiunti

\end{itemize}

\sphinxAtStartPar
\sphinxstylestrong{Argomenti.}
\begin{itemize}
\item {} 
\sphinxAtStartPar
valorizzazione del progresso e della scienza, come unica forma di conoscenza valida, attraverso il metodo scientifico;

\item {} 
\sphinxAtStartPar
estensione del metodo scientifico ad ambiti sociali, come in logica, psicologia, sociologia, economia

\item {} 
\sphinxAtStartPar
rifiuto della metafisica, poiché metodo scientifico inapplicabile

\item {} 
\sphinxAtStartPar
a cosa serve la filosofia? Necessaria riforma: abbandonare studi di metafisica, abbracciare tutti gli argomenti trattati se applicabile il metodo scientifico; una volta che una disciplina è una disciplina scientifica, non ci sono discipline privilegiate. La filosofia come metodi comuni a tutte le scienze, attività di coordinamento e guida, in un desiderio di unificazione.

\end{itemize}

\sphinxAtStartPar
Due periodi:
\begin{itemize}
\item {} 
\sphinxAtStartPar
primo positivismo: sviluppatosi in Francia, Comte. Subisce ancora la forte influenza dell’idealismo. Filosofia progressista dal punto di vista scientifico, conservatrice dal punto di vista politico (contraria ai moti)

\item {} 
\sphinxAtStartPar
secondo positivismo: progressismo dal punto di vista scientifico e politico, filosofia liberale, che caratterizza il ceto borghese

\end{itemize}

\sphinxAtStartPar
Altri aspetti
\begin{itemize}
\item {} 
\sphinxAtStartPar
positivismo sociale: Comte, sociologia

\item {} 
\sphinxAtStartPar
positivismo (metodo)logico: Stuart Mill, si concentra sul metodo per fare scienza e conoscenza

\item {} 
\sphinxAtStartPar
positivismo evoluzionistico: influenzato da Darwin, filosofi prenderanno le osservazioni di Darwin sull’evoluzione per farne una legge di natura e applicarla agli ambiti della filosofia

\end{itemize}

\sphinxAtStartPar
\sphinxstylestrong{Personaggi.}
\begin{itemize}
\item {} 
\sphinxAtStartPar
Comte

\item {} 
\sphinxAtStartPar
\sphinxstyleemphasis{Darwin}

\item {} 
\sphinxAtStartPar
Stuart Mill

\end{itemize}

\sphinxstepscope


\chapter{Argomenti}
\label{\detokenize{ch/topics:argomenti}}\label{\detokenize{ch/topics:philosophy-topics}}\label{\detokenize{ch/topics::doc}}

\section{Ontologia}
\label{\detokenize{ch/topics:ontologia}}\begin{itemize}
\item {} 
\sphinxAtStartPar
Kant:

\end{itemize}


\section{Metafisica}
\label{\detokenize{ch/topics:metafisica}}

\section{}
\label{\detokenize{ch/topics:id1}}






\renewcommand{\indexname}{Indice}
\printindex
\end{document}