%% Generated by Sphinx.
\def\sphinxdocclass{jupyterBook}
\documentclass[letterpaper,10pt,italian]{jupyterBook}
\ifdefined\pdfpxdimen
   \let\sphinxpxdimen\pdfpxdimen\else\newdimen\sphinxpxdimen
\fi \sphinxpxdimen=.75bp\relax
\ifdefined\pdfimageresolution
    \pdfimageresolution= \numexpr \dimexpr1in\relax/\sphinxpxdimen\relax
\fi
%% let collapsible pdf bookmarks panel have high depth per default
\PassOptionsToPackage{bookmarksdepth=5}{hyperref}
%% turn off hyperref patch of \index as sphinx.xdy xindy module takes care of
%% suitable \hyperpage mark-up, working around hyperref-xindy incompatibility
\PassOptionsToPackage{hyperindex=false}{hyperref}
%% memoir class requires extra handling
\makeatletter\@ifclassloaded{memoir}
{\ifdefined\memhyperindexfalse\memhyperindexfalse\fi}{}\makeatother

\PassOptionsToPackage{warn}{textcomp}

\catcode`^^^^00a0\active\protected\def^^^^00a0{\leavevmode\nobreak\ }
\usepackage{cmap}
\usepackage{fontspec}
\defaultfontfeatures[\rmfamily,\sffamily,\ttfamily]{}
\usepackage{amsmath,amssymb,amstext}
\usepackage{polyglossia}
\setmainlanguage{italian}



\setmainfont{FreeSerif}[
  Extension      = .otf,
  UprightFont    = *,
  ItalicFont     = *Italic,
  BoldFont       = *Bold,
  BoldItalicFont = *BoldItalic
]
\setsansfont{FreeSans}[
  Extension      = .otf,
  UprightFont    = *,
  ItalicFont     = *Oblique,
  BoldFont       = *Bold,
  BoldItalicFont = *BoldOblique,
]
\setmonofont{FreeMono}[
  Extension      = .otf,
  UprightFont    = *,
  ItalicFont     = *Oblique,
  BoldFont       = *Bold,
  BoldItalicFont = *BoldOblique,
]



\usepackage[Sonny]{fncychap}
\ChNameVar{\Large\normalfont\sffamily}
\ChTitleVar{\Large\normalfont\sffamily}
\usepackage[,numfigreset=1,mathnumfig]{sphinx}

\fvset{fontsize=\small}
\usepackage{geometry}


% Include hyperref last.
\usepackage{hyperref}
% Fix anchor placement for figures with captions.
\usepackage{hypcap}% it must be loaded after hyperref.
% Set up styles of URL: it should be placed after hyperref.
\urlstyle{same}


\usepackage{sphinxmessages}



        % Start of preamble defined in sphinx-jupyterbook-latex %
         \usepackage[Latin,Greek]{ucharclasses}
        \usepackage{unicode-math}
        % fixing title of the toc
        \addto\captionsenglish{\renewcommand{\contentsname}{Contents}}
        \hypersetup{
            pdfencoding=auto,
            psdextra
        }
        % End of preamble defined in sphinx-jupyterbook-latex %
        

\title{basics book template}
\date{21 dic 2024}
\release{}
\author{basics}
\newcommand{\sphinxlogo}{\vbox{}}
\renewcommand{\releasename}{}
\makeindex
\begin{document}

\pagestyle{empty}
\sphinxmaketitle
\pagestyle{plain}
\sphinxtableofcontents
\pagestyle{normal}
\phantomsection\label{\detokenize{intro::doc}}


\sphinxAtStartPar
If you want ot start a new basics\sphinxhyphen{}book, it could be a good idea to start from this template.

\sphinxAtStartPar
Please check out the Github repo of the project, \sphinxhref{https://github.com/Basics2022}{basics\sphinxhyphen{}book project}.
\begin{itemize}
\item {} 
\sphinxAtStartPar
{\hyperref[\detokenize{ch/history::doc}]{\sphinxcrossref{Cronologia}}}

\item {} 
\sphinxAtStartPar
{\hyperref[\detokenize{ch/topics::doc}]{\sphinxcrossref{Argomenti}}}

\end{itemize}

\sphinxstepscope


\chapter{Cronologia}
\label{\detokenize{ch/history:cronologia}}\label{\detokenize{ch/history:philosophy-chronology}}\label{\detokenize{ch/history::doc}}

\section{Illuminismo}
\label{\detokenize{ch/history:illuminismo}}\label{\detokenize{ch/history:pc-illuminism}}
\sphinxAtStartPar
\sphinxstylestrong{Personaggi.}
\begin{itemize}
\item {} 
\sphinxAtStartPar
Kant(Konigsberg, 1724\sphinxhyphen{}1804): critiche

\end{itemize}


\section{Idealismo}
\label{\detokenize{ch/history:pc-idealism}}\label{\detokenize{ch/history:idealismo}}

\section{Positivismo}
\label{\detokenize{ch/history:positivismo}}\label{\detokenize{ch/history:pc-positivism}}
\sphinxAtStartPar
\sphinxstylestrong{Contesto.} Sviluppo nella prima metà dell’XIX secolo in Francia e Gran Bretagna, diffusione nella seconda metà del secolo. Il secolo è caratterizzato dalla rivoluzione industriale.

\sphinxAtStartPar
\sphinxstylestrong{Legami.}
\begin{itemize}
\item {} 
\sphinxAtStartPar
Illuminismo: fiducia nella scienza; ma illuminismo critico nei confronti dell’esistente, gran differenza tra il mondo com’è e il mondo come dovrebbe essere (Kant, Voltaire, Rousseau); i filosofi positivisti invece sono allineati al pensiero dominante del tempo: la prima parte del positivismo è conservatrice, alllineata con i sovrani assoluti; la seconda parte è espressione della classe borghese, come classe dominante: pensiero dei risultati raggiunti

\end{itemize}

\sphinxAtStartPar
\sphinxstylestrong{Argomenti.}
\begin{itemize}
\item {} 
\sphinxAtStartPar
valorizzazione del progresso e della scienza, come unica forma di conoscenza valida, attraverso il metodo scientifico;

\item {} 
\sphinxAtStartPar
estensione del metodo scientifico ad ambiti sociali, come in logica, psicologia, sociologia, economia

\item {} 
\sphinxAtStartPar
rifiuto della metafisica, poiché metodo scientifico inapplicabile

\item {} 
\sphinxAtStartPar
a cosa serve la filosofia? Necessaria riforma: abbandonare studi di metafisica, abbracciare tutti gli argomenti trattati se applicabile il metodo scientifico; una volta che una disciplina è una disciplina scientifica, non ci sono discipline privilegiate. La filosofia come metodi comuni a tutte le scienze, attività di coordinamento e guida, in un desiderio di unificazione.

\end{itemize}

\sphinxAtStartPar
Due periodi:
\begin{itemize}
\item {} 
\sphinxAtStartPar
primo positivismo: sviluppatosi in Francia, Comte. Subisce ancora la forte influenza dell’idealismo. Filosofia progressista dal punto di vista scientifico, conservatrice dal punto di vista politico (contraria ai moti)

\item {} 
\sphinxAtStartPar
secondo positivismo: progressismo dal punto di vista scientifico e politico, filosofia liberale, che caratterizza il ceto borghese

\end{itemize}

\sphinxAtStartPar
Altri aspetti
\begin{itemize}
\item {} 
\sphinxAtStartPar
positivismo sociale: Comte, sociologia

\item {} 
\sphinxAtStartPar
positivismo (metodo)logico: Stuart Mill, si concentra sul metodo per fare scienza e conoscenza

\item {} 
\sphinxAtStartPar
positivismo evoluzionistico: influenzato da Darwin, filosofi prenderanno le osservazioni di Darwin sull’evoluzione per farne una legge di natura e applicarla agli ambiti della filosofia

\end{itemize}

\sphinxAtStartPar
\sphinxstylestrong{Personaggi.}

\phantomsection\label{\detokenize{ch/history:pc-saint-simon}}\subsubsection*{Saint\sphinxhyphen{}Simon (Parigi, )}

\sphinxAtStartPar
\sphinxstylestrong{Biografia.}
\sphinxstylestrong{Opere.}
\sphinxstylestrong{Pensiero.}
\begin{itemize}
\item {} 
\sphinxAtStartPar
Storia tende naturalmente al progresso, ma segnata da alternanza di \sphinxstyleemphasis{epoche critiche} ed \sphinxstyleemphasis{epoche organiche}. A inizio “800 è finita l’epoca critica della nascita di un nuova scienza e della rivoluzione industriale, ora che si sono imposte la storia e l’umanità può evolvere in maniera costante

\item {} 
\sphinxAtStartPar
Scontro tra ceti produttivi e parassiti, i ceti improduttivi: padroni e operai dovrebbero allearsi contro nobili, clero, burocrati che vivono sulle spalle dei lavoratori

\end{itemize}
\phantomsection\label{\detokenize{ch/history:pc-comte}}\subsubsection*{Comte}

\sphinxAtStartPar
\sphinxstylestrong{Biografia} Allievo (\sphinxstylestrong{todo} \sphinxstyleemphasis{di che?}) di {\hyperref[\detokenize{ch/history:pc-saint-simon}]{\sphinxcrossref{\DUrole{std,std-ref}{Saint\sphinxhyphen{}Simon}}}}. Crollo psicologico…
\sphinxstylestrong{Opere.} 1830, \sphinxstyleemphasis{Corso di filosofia positiva}
\sphinxstylestrong{Pensiero.}
\begin{itemize}
\item {} 
\sphinxAtStartPar
La scienza ha guidato la storia e può continuare a guidare la storia. Il metodo scientifico può essere applicato a tutti gli ambiti della vita.

\item {} 
\sphinxAtStartPar
\sphinxstylestrong{Legge dei tre stadi.} Attraverso lo studio della storia delle scienza (astronomia, fisica\sphinxhyphen{}meccanica, chimica, biologia) propone una legge di evoluzione delle scienze attraverso tre stadi:
\begin{enumerate}
\sphinxsetlistlabels{\arabic}{enumi}{enumii}{}{.}%
\item {} 
\sphinxAtStartPar
stadio teologico o fittizio: l’uomo usa l’immaginazione per dare una spiegazione tramite entità immaginarie soprannaturali

\item {} 
\sphinxAtStartPar
stadio metafisico: astrazioni personificate

\item {} 
\sphinxAtStartPar
stadio positivo o scientifico: conoscenze tramite fatti e relazioni

\end{enumerate}

\item {} 
\sphinxAtStartPar
C’è ancora una disciplina che non ha ancora compiuto l’ultimo passo: la \sphinxstylestrong{sociologia}. Necessità di compiere questo passo per comprendere, \sphinxstylestrong{prevedere} e controllare le dinamiche della società, ad esempio per evitare crisi. Visto che queste leggi non esistono, si procede «a tentoni». Lo scopo della filosofia di Comte è portare la sociologia a compiere questo ultimo passo.

\item {} 
\sphinxAtStartPar
Mancano all’appello 3 discipline: matematica (linguaggio della scienza), logica (grammatica della scienza), psicologia. La psicologia deve essere divisa in due parti: gli argomenti sono riconducibili o alla biologia o alla sociologia.

\item {} 
\sphinxAtStartPar
\sphinxstylestrong{Sociologia.} Comte distingue due discipline
\begin{itemize}
\item {} 
\sphinxAtStartPar
statica sociale: fotografia dell’istante; a ogni sistema politico corrisponde un sistema economico e sociale; la statica si occupa dell”\sphinxstylestrong{ordine} di un tipo di società

\item {} 
\sphinxAtStartPar
dinamica sociale: cambiamenti, \sphinxstylestrong{progresso}; per Comte la \sphinxstylestrong{storia} è una evoluzione che segue un \sphinxstylestrong{percorso necessario}, come Hegel
\begin{itemize}
\item {} 
\sphinxAtStartPar
per Hegel la storia è guidata dalla provvidenza, l’assoluto, una divinità spirituale; per Comte la storia è guidata dalla scienza

\item {} 
\sphinxAtStartPar
chi porta avanti la scienza: uomini di genio; per Hegel, eroi o veggenti; per Comte, gli uomini di scienza

\end{itemize}

\end{itemize}

\end{itemize}
\subsubsection*{Darwin}
\subsubsection*{Stuart Mill}

\sphinxstepscope


\chapter{Argomenti}
\label{\detokenize{ch/topics:argomenti}}\label{\detokenize{ch/topics:philosophy-topics}}\label{\detokenize{ch/topics::doc}}

\section{Ontologia}
\label{\detokenize{ch/topics:ontologia}}\begin{itemize}
\item {} 
\sphinxAtStartPar
Kant:

\item {} 
\sphinxAtStartPar
{\hyperref[\detokenize{ch/history:pc-comte}]{\sphinxcrossref{\DUrole{std,std-ref}{Comte}}}}: legge dei tre stadi

\end{itemize}


\section{Metafisica}
\label{\detokenize{ch/topics:metafisica}}

\section{Storia e sociologia}
\label{\detokenize{ch/topics:storia-e-sociologia}}\begin{itemize}
\item {} 
\sphinxAtStartPar
\DUrole{xref,myst}{Hegel}

\item {} 
\sphinxAtStartPar
{\hyperref[\detokenize{ch/history:pc-saint-simon}]{\sphinxcrossref{\DUrole{std,std-ref}{Saint\sphinxhyphen{}Simon}}}}

\item {} 
\sphinxAtStartPar
{\hyperref[\detokenize{ch/history:pc-comte}]{\sphinxcrossref{\DUrole{std,std-ref}{Comte}}}}

\end{itemize}


\section{Psicologia}
\label{\detokenize{ch/topics:psicologia}}\begin{itemize}
\item {} 
\sphinxAtStartPar
{\hyperref[\detokenize{ch/history:pc-comte}]{\sphinxcrossref{\DUrole{std,std-ref}{Comte}}}}: psicologia da smembrare e ricondurre a biologia o a \sphinxstylestrong{sociologia}

\end{itemize}







\renewcommand{\indexname}{Indice}
\printindex
\end{document}